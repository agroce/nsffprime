\section{Problem Statement, Overview, and Objectives}

%\subsection{Problem Statement}
%\input{problem}

Developing complex embedded systems software has always posed unusual
challenges:  because such systems tend to directly impact the physical
world, they often unusually stringent correctness requirements.
Conventional software often has the recourse to ``restore to backup''
or simply reboot; embedded systems software that interacts with
actuators in addition to sensors constantly faces the problem of
irrevocable damage to physical systems for which no backup is
possible.  In the extreme case, such as an embedded system operating
on a probe orbiting Saturn or a rover on Mars, even post-mortem
analysis, much less diagnosis and repair, may be impossible after a software error.

On the other hand, such systems often face limitations in terms of
resources:  hardware, not software, development is often the core
concern of an embedded systems project.  The exact portion of the NASA
Curiosity Mars Rover budget devoted to software is difficult to
ascertain, but it certainly amounts to less than 25\% of the total
Rover budget, despite being a monumental software
project~\cite{NASAbudget}.  The actual likely value is somewhere below
10\%, since the 25\% figure includes all R\&D, design, and spacecraft
construction costs.

Building software with extremely rigorous correctness requirements,
often under tight budget constraints relative to total project budget,
and almost always in low-level languages, using error-prone memory
allocation methods, where concurrency with real-time constraints is
also almost always involved, poses a serious challenge indeed.

One way embedded developers meet such challenges is by making use of
\emph{embedded systems development frameworks:} collections of tools,
libraries, and processes that streamline the creation of software for
embedded devices.  Just as web developers make use of web development
frameworks, embedded developers make use of pre-built functionality beyond the capabilities provided
by C and C++ standard libraries and conventional real-time operating
systems such as VxWorks in the form of systems that constrain but ease
development of the kinds of systems they build.  A key difference
between, e.g., web development frameworks and embedded systems
frameworks, however, is that some of the most prominent embedded
frameworks have a major focus on correctness and reliability, rather
than simply ease of development.  The high cost of embedded failures
is to some extent represented in the nature of embedded frameworks.

However, by and large, this focus on specification, correctness, and
reliability is not connected to state-of-the-art testing methods.
Embedded systems failures due to very rare circumstances can be
catastrophic, but widely-used embedded systems frameworks do not
facilitate testing beyond standard manual unit and integration
testing methods.  Particularly in concurrent software, or software
involving manipulation of bit-level data, however, human ingenuity is
hard-pressed to explore the variety of inputs and hardware failure scenarios that
embedded systems must handle.  Modern test generation techniques,
particularly fuzzing, but also including symbolic/concolic execution
and model checking, are natural fits for these kinds of
(semi-)exhaustive input explorations, but are 1) not part of the
reportoire of most embedded systems developers or test engineers and
2) not always easily adaptable to embedded contexts.  This project
aims to address this problem by  developing methods and tools for
\emph{framework-aware fuzzing and lightweight verification}, using
NASA's FPrime embedded development framework as a central testbed.
The approach proposed exploits framework-provided constraints and type
information to generate test harnesses and guide \emph{semantic-aware
  fuzzing}. With \textbf{DeepState} as the primary interface and
exploratory integration with C++-capable model checkers, this project combines fuzzing with verification to explore \emph{meaningful} behaviors automatically. Framework metadata provides semantic constraints that focus fuzzers on valid inputs; when data violates preconditions, \emph{precondition mapping} transforms it into satisfying assignments.
The work also introduces \emph{framework-customized mutation testing},
defining operators aligned with FPrime’s architecture to guide and
evaluate testing. Methods will be demonstrated on FPrime deployments
and generalized to other frameworks.   The end-goal of this project is
to allow embedded systems developers to employ state-of-the-art
methods for finding bugs in software systems, in ways that
automatically leverage the additional information provided by the use
of embedded systems frameworks, while remaining within the paradigms
and development practices with which they are familliar.

\subsection{NASA's FPrime Framework}

\section{Intellectual Merit}

\subsection{PI Qualifications}

PI Groce has a long history with software analysis and verification,
including involvement in design and development of well-known model
checkers and fuzzing tools, and application of model checkers at
NASA/JPL on flight software for the Mars Rovers, including application
of and development contributions to, the SPIN and CBMC model checkers,
the TSTL testing tool for Python, and the DeepState unit-fuzzing
interface for C++ and C code.  Groce was the lead for design and
development of automated testing tools and specification languages for
the Mars Science Laboratory (Curiosity Rover) project.  He received
his PhD under the late Turing laureate Edmund M. Clarke, Jr., as one
of the first PhD graduates focusing on model checking for software
using SAT and SMT solvers.  More recently, he has primarily focused on
developing algorithms and tools for automated software testing; he is
the lead developer of the DeepState, universalmutator, and TSTL design and development teams.
Groce is the lead developer for multiple GitHub-hosted testing and
verification tools (including DeepState, the focus platform for this
project), all with more than 100 GitHub ``stars''; he reports 50-100
bugs detected using such tools to open source projects each year, and
has been awarded significant bug bounties for security-critical
compiler bugs detected using his work.  Groce was recently employed by
the Bitcoin Core team to investigate the effectiveness of their
fuzzing framework, a key component in ensuring the security of the
leading implementation of the most important cryptocurrency.  His work
has consistently invovled a marriage of deep theory of testing and
analysis with practical work oriented towards use by real-world
developers and implemented in working tools.


\subsection{}