\section{Broader Impacts}

\paragraph{Improving Software System Reliability:} A key element of
this proposal's approach is to focus on realistically deployable
techniques.  We initially aim
for strong integration with NASA's FPrime~\cite{fprime,fprimerepo}
open source
flight software architecture and platform; PI Groce is already in
discussion with engineers at NASA's Jet Propulsion Laboratory, and
engaged in producing tests for the FPrime autocoder using DeepState,
and has previously sent summer students to work with NASA's FPrime team.
This integration will allow our
methods to be applied to CubeSat missions, instruments, and other flight software
systems, leading to improved reliability for low-budget space-based
scientific efforts.  We expect, in the long run, that our approaches
will lead to more reliable and robust development in many embedded and
cyberphysical systems domains.

\paragraph{Education and Outreach:}
The proposed research yields several opportunities for enhancing CS
education, recruiting new CS majors, and retaining CS students.  PI Groce will work with the NAU Student ACM Chapter to
present a series of ``excursions in testing'' that introduce automated
testing to students, using DeepState to find bugs in real world code,
including code from media player libraries.  The
work of Guzdial \cite{Guzdial} has shown that media computation is an effective way to both recruit and retainstudents in computer science. Groce is also teaching
a class on automated testing of embedded systems.  Groce's prior
engagement in teaching and mentoring high school students (including
as summer students who have published their work in major venues) and working with
local technically-oriented 4H chapters, including in rural communities
(such as Williams, AZ) surrounding Flagstaff will also allow
preparation and presentation of versions of these excursions tailored
to high school students' level of knowledge.